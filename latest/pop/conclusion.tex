%\section{Conclusion}
%\label{sec:conc}
In summary, 
we have analyzed adiabaticity in magnetospheric chorus waves using test particle simulations. By synchronizing with the real-time chirping wave, we calculate the adiabatic invariant, revealing variations across reference frames and regions. 
%our analysis of adiabaticity in magnetospheric chorus waves relies on test particle simulations to compute the adiabatic invariant, revealing variations in trapped particles' adiabaticity across reference frames and spatial regions. 
Near the equator, adiabaticity is compromised due to particle release. 
%Notably, proximity to the equator compromises adiabaticity. 
Our findings are further validated through a reduced model of energetic trapped particle current and Vlasov simulations, affirming the validity of the adiabatic description and shedding light on wave chirping behavior relative to the inhomogeneity parameter $\alpha$.


%Finally, we validate our findings with a simplified model of energetic trapped particle current and Vlasov simulation.