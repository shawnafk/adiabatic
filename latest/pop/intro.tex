%\section{Introduction}}
%\label{sec:intro}
%preface wave instabilitys with frequency chirping: fusion and magnetoshpere
Wave-particle resonant interactions play an important role  in a wide range of wave instabilities
%, among which the most representative branch is that occurs 
with frequency chirping in fusion and magnetoshperic plasmas.
In the fusion related plasmas, the frequency chirping often appears with the Alfv\'en wave instabilities \cite{chen2016,wang2018,wang2012,wang2012a} and energetic-particle-driven geodesic acoustic modes \cite{wang2013}.
The wave chirping is usually accompanied by fast ion loss and strong radial transport, which is greatly concerned in the fusion studies.
In the planetary magnetosphere, there exists the chorus wave, which is a peculiar electromagnetic emission with frequency sweeping in the whistler-mode range of frequencies \cite{helliwell1965whistlers,burtis_magnetospheric_1976,tsurutani_postmidnight_1974}. 
The chirping behavior similar to the magnetospheric chorus has also been studied recently in the laboratory plasmas \cite{vancompernolle2015,saitoh2024}.
The chorus waves  play a critical role in the electron acceleration   in the Van Allen radiation belts \cite{horne_wave_2005,thorne_rapid_2013,reeves_electron_2013} and the pulsating aurora  and diffuse auroral precipitation \cite{nishimura_identifying_2010,kasahara_pulsating_2018,thorne_scattering_2010}.
The chorus emission is also related to 
 %which have received a significant research interest, such as in 
 the studies of the superradiance in free-electron lasers \cite{zonca_nonlinear_2021, soto-chavez2012}.

%1. The wave particle interaction as pendulum Hamiltonain Albert
The physics of the chorus wave in essence are the nonlinear wave-particle interactions \cite{oneil1971,oneil1972} between the resonant trapped electrons and the whistler waves \cite{omura_theory_2008, an2019}.
The Hamiltonian treatment of a particle nearly resonant with a monochromatic whistler wave in the inhomogeneous magnetic field has been extensively studied \cite{albert1993,albert2021}.
The particle dynamics can be described by a modified pendulum Hamiltonian, where the trapped particles feel both the wave restoring force and the background drag force related to the frequency variation and magnetic field gradient \cite{zheng2024,tao_trap-release-amplify_2021}.

%What is different for the chirping wave interaction, and what changes
It is worth noting that the parameters in the Hamiltonian are changing as the frequency varying together with the inhomogeneous magnetic field.
Similar to the pendulum system, if the variation of the parameters is slow compared to the trapped particle's bounce motion, then the system is adiabatic, and there exists an adiabatic invariant associated with the bounce motion.
%adiabaitic: the definiation and judge know the adiabaticity is of imporatant to understand and simplify the question
%argument for the interaction regime and adiabatic, it should not be adiabatic: tao and Bierwage
However, recent studies suggest that in the case of the chorus frequency chirping phenomenon, the nonlinear chirping timescale $\tau_{nl}$ is comparable to the bounce period $\tau_{b}$. 
Consequently, the interaction becomes nonadiabatic \cite{tao2017a,tao2017b}.

Yet, assessing whether the resonant particles undergo adiabatic motion solely based on a comparison between the chirping and bounce scales is overly simplistic.
Firstly, the adiabatic invariant governing the resonant interaction of trapped particles with chirping chorus waves in the magnetosphere has never been clearly defined, let alone violated.
Furthermore, during the triggering and propagation of chorus waves, particles undergo distinct phases of wave-particle interactions depending on their location and the stages of interaction.
In the early chirping phase \cite{bierwage2021}, adiabatic bounce motion is invalid, as the trapped state has yet to be established.
Moreover, in the wave triggering region, particle release \cite{tao_trap-release-amplify_2021} often accompanies fast wave variations, constituting a nonadiabatic process.
Nevertheless, there also exists stable trapped behavior \cite{zheng2024} in the propagation region of the chorus away from the magnetic equator, where adiabaticity is poorly understood in current research. 
Establishing stringent criteria for particle adiabaticity is paramount.
Not only does it aid in identifying key interaction stages and understanding dynamic shifts between them, but it also furnishes additional approximations to simplify and solve chorus evolution in the propagation region.

In our previous work, we developed a novel Hamiltonian theory \cite{zheng2024} specifically tailored to address frequency chirping chorus waves.
By utilizing the local resonance frame of reference, we derived a reduced Hamiltonian that accurately depicts the interaction between energetic particles and the slowly varying envelope of the waves in the magnetosphere.
Focusing specifically on the onset stage of chirping, we developed a corresponding  $\delta f$ Vlasov simulation method and successfully replicated the generation of chirping chorus waves.
Recognizing the critical role of the background magnetic field in the chorus wave-particle interaction process \cite{wu2023,wu_controlling_2020}, our simulation incorporated a realistic Earth dipole field configuration.
The chirping rate and amplitude observed in our simulation closely match satellite observations \cite{cully_observational_2011}. 
Utilizing this simulated self-consistent slowly varying wave envelope, we can easily conduct test particle simulations, providing a unique perspective distinct from conventional methods.
While test particle methods \cite{huanghua_test_ptc,omura_test_ptc,tao_test_ptc} offer an intuitive approach to studying particle dynamics within a wave field, conventional studies often analyze particle dynamics within the laboratory reference frame. In this frame, the Hamiltonian incorporates  a rapidly varying phase term to model the wave-particle interaction, which complicates the assessment of adiabaticity. However, by reconsidering the interaction within a reference frame perfectly synchronized with the resonance—moving in-time with the chirping wave that traps the resonant particle—the fast-varying scale is mitigated. Consequently, a particle can be considered adiabatic as long as it remains trapped  \cite{CARY1989287}.

In this letter, we have explored the dynamics of resonant electrons and the evolution of whistler-mode chorus waves by test particle simulations.
We show that the adiabatic description is valid if one tracks the dynamics in the frame of reference moving the local resonance.
We have derived an analytic current expression and discussed the adiabatic regime in the global domain by showing the wave chirping behavior with respect to the inhomogeneity parameter $\alpha$.
The adiabatic approximation and the reduced current expression reveal the underlying physics for the wave chirping.


%In this work, we examine the adiabaticity by the test particle simulation and derive the nonlinear current with adiabatic approximation for 
%the magnetospheric chorus.
%We introduce a canonical transformation that shifts the static frame to a time-independent reference frame which follows with the real time chirping wave in the Vlasov simulation.
%For the interaction of chorus waves and energetic electrons,
%we solve the Hamilton's equations and calculate the adiabatic invariant from the test particle simulations.
%The energetic electrons move opposite to the motion of the chorus wave packet through several distinct regions with different physical properties.
%We show that the adiabaticity can be completely different in the different reference frames and in the different regions  along the Earth's dipole magnetic fields. 
%In the resonance frame,  the adiabatic invariant is preserved in the most of the  downstream region, where the   chorus waves  propagate with frequency chirping and grow in amplitudes.
%As the energetic electrons approach the equator
%where the chorus waves are triggered with initial small amplitudes,
%the adiabaticity  is no longer valid in the equator and the upstream region  due to the particle releasing.
%From the adiabatic invariant, we further obtain a reduced  form of the energetic trapped particle current.
%We verify the adiabatic approximation from the nonlinear current calculated in the Vlasov simulation and show that the adiabatic regime is valid in the most of the wave propagation region, except in the source regions where the wave is generated from.

%We organize our 
%The paper is organized as follows. In section~\ref{sec:theory}, we review the Hamiltonian theory in the resonance reference frame. In section~\ref{sec:dis},  
%the adiabaticity 
%is examined by the test particle simulation and  the nonlinear current 
%is derived with adiabatic approximation. Finally, the summary is presented in section~\ref{sec:conc}.