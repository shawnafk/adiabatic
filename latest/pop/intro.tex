\section{Introduction}
\label{sec:intro}
%preface wave instabilitys with frequency chirping: fusion and magnetoshpere
Wave-particle resonant interactions play an important role  in a wide range of wave instabilities
%, among which the most representative branch is that occurs 
with frequency chirping in fusion and magnetoshperic plasmas.
In the fusion related plasmas, the frequency chirping often appears with the Alfv\'en wave instabilities \cite{chen2016,wang2018,wang2012,wang2012a} and energetic-particle-driven geodesic acoustic modes \cite{wang2013}.
The wave chirping is usually accompanied by fast ion loss and strong radial transport, which is greatly concerned in the fusion studies.
In the planetary magnetosphere, there exists the chorus wave, which is a peculiar electromagnetic emission with frequency sweeping in the whistler-mode range of frequencies \cite{helliwell1965whistlers,burtis_magnetospheric_1976,tsurutani_postmidnight_1974}. 
The chirping behavior similar to the magnetospheric chorus has also been studied recently in the laboratory plasmas \cite{vancompernolle2015,saitoh2024}.
The chorus waves  play a critical role in the electron acceleration   in the Van Allen radiation belts \cite{horne_wave_2005,thorne_rapid_2013,reeves_electron_2013} and the pulsating aurora  and diffuse auroral precipitation \cite{nishimura_identifying_2010,kasahara_pulsating_2018,thorne_scattering_2010}.
The chorus emission is also related to 
 %which have received a significant research interest, such as in 
 the studies of the superradiance in free-electron lasers \cite{zonca_nonlinear_2021, soto-chavez2012}.

%1. The wave particle interaction as pendulum Hamiltonain Albert
The physics of the chorus wave in essence are the nonlinear wave-particle interactions \cite{oneil1971,oneil1972} between the resonant trapped electrons and the whistler waves \cite{omura_theory_2008, an2019}.
The Hamiltonian treatment of particle nearly resonant with a monochromatic whistler wave in the inhomogeneous magnetic field has been extensively studied \cite{albert1993,albert2021}.
The particle dynamics can be described by a modified pendulum Hamiltonian, where the trapped particles feel both the wave restoring force and the background drag force related to the frequency variation and magnetic field gradient \cite{zheng2024,tao_trap-release-amplify_2021}.

%What is different for the chirping wave interaction, and what changes
It is worth noting that the parameters in the Hamiltonian are changing as the frequency varying together with the inhomogeneous magnetic field.
Alike to the pendulum system, if the variation of the parameters are slow compare to the trapped particle bounce motion, then the system is adiabatic and there exists an adiabatic invariant associated with the bounce motion.
%adiabaitic: the definiation and judge know the adiabaticity is of imporatant to understand and simplify the question
%argument for the interaction regime and adiabatic, it should not be adiabatic: tao and Bierwage
However, recently studies state that for the chorus frequency chirping problem, the nonlinear chirping timescale $\tau_{nl}$ is comparable to the bounce period $\tau_{b}$. Thus, the interaction is nonadiabatic \cite{tao2017a,tao2017b}. 
%0413 + motivation
However, the judgement on whether the resonance particle go through an adiabatic motion just compare the chirping and bounce scale is too rough.
Firstly, the adiabatic invariant for the trapped particle resonantly interacts with chirping chorus wave in the magnetosphere have never been clear defined, not to mention the violation of such.
Besides, during the triggering and propagation of chorus wave, the particle endures multiple phases of wave-particle interactions depending on the location and time stage of the interaction.
In the early phase of chirping \cite{bierwage2021}, the adiabatic bounce motion does not valid since the trapped have yet to be established.
Also in the wave triggering region, there usually accompany by particle releasing \cite{tao_trap-release-amplify_2021} and fast wave variation which is also a nonadiabatic process.
However, there also exists a stable trapped behavior \cite{zheng2024} in the propagation region of the chorus off the magnetic equator, where the aidabaticity is not well understand in current studies. 
%we we have done, first: frame of reference, then definition of I
%Understanding the dynamics during the entire wave-particle interaction process  Reveal to the restrict criterion for the particle adiabaticity is of great importance. 
Not only does it help us to distinguish the key stages of the interaction and understand how the dynamics shift in between, it also provides additional approximations that can be applied to simplify and solve the chorus evolution in the propagation region.

The test particle methods \cite{huanghua_test_ptc,omura_test_ptc,tao_test_ptc} provides intuitive wave to study the particle dynamics under the wave field.
However, the conventional studies usually manage the particles' dynamics in the laboratory reference frame,
%people would see fast varying particle and wave phase, There exists fast varing phase in the interaction Hamiltonian 
in which, the wave vector undergoes accelerated rotation due to frequency chirping and the Hamiltonian involves a fast varying phase term in the wave-particle interaction part. 
Thus, it is hard to tell the adiabaticity in such description.
%However, resonance could obtain ....
However, if we reconsider the interaction on a reference frame that perfectly follows the resonance, i.e., moving in-time with the chirping wave that traps the resonant particle, we would observe no chirping at all. Thus, the fast-varying scale is mitigated and a particle can be regarded as adiabatic as long as it remains trapped. Only the particle near the separatrix region suffers a nonadiabatic motion, as pointed in previous studies \cite{CARY1989287}.

In our previous work, we have developed a novel Hamiltonian theory \cite{zheng2024}. 
Harnessing the local resonance frame of reference, we derived the reduced Hamiltonian that describes the interaction of the particle with the slowly varying envelope.
With a further restriction to the onset stage of the chirping, we have constructed a corresponding $\delta f$ Vlasov simulation method and successful obtained the generation of chirping chorus wave.
Note that, since the background magnetic field can strongly affect the wave-particle interaction process \cite{wu2023,wu_controlling_2020}, our simulation is carried with real Earth dipole field configuration.
The chirping rate and amplitude are well agreement with satellite observation \cite{cully_observational_2011} and PIC simulations with real field parameters \cite{katoh2016}.
%Also, since we choose a static frame that follow the start point of the chirping instead of tracking the chirping with time, the simulation obtains the chirping effect as a slowly varying modulation on the wave envelope. 
The simulated self-consistent slowly varying wave envelope can be readily applied to perform a test particle simulation to study the particle dynamics in a new angle of view different with conventional test particle methods. 

In this work, we examine the adiabaticity by the test particle simulation and derive the nonlinear current with adiabatic approximation for 
the magnetospheric chorus.
We introduce a canonical transformation that shifts the static frame to a time-independent reference frame which follows with the real time chirping wave in the Vlasov simulation.
For the interaction of chorus waves and energetic electrons,
we solve the Hamilton's equations and calculate the adiabatic invariant from the test particle simulations.
The energetic electrons move opposite to the motion of the chorus wave packet through several distinct regions with different physical properties.
We show that the adiabaticity can be completely different in the different reference frames and in the different regions  along the Earth's dipole magnetic fields. 
In the resonance frame,  the adiabatic invariant is preserved in the most of the  downstream region where the   chorus waves  propagate with frequency chirping and grow in amplitudes.
As the energetic electrons approach the equator
where the chorus waves are triggered with initial small amplitudes,
the adiabaticity  is no longer valid in the equator and the upstream region  due to the particle releasing.
From the adiabatic invariant, we further obtain a reduced  form of the energetic trapped particle current.
We verify the adiabatic approximation from the nonlinear current calculated in the Vlasov simulation and show that the adiabatic regime is valid in the most of downstream region, except in the upstream and equator regions.

%We organize our 
The paper is organized as follows. In section~\ref{sec:theory}, we review the Hamiltonian theory in the resonance reference frame. In section~\ref{sec:dis},  
the adiabaticity 
is examined by the test particle simulation and  the nonlinear current 
is derived with adiabatic approximation. Finally, the summary is presented in section~\ref{sec:conc}.
