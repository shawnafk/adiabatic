%\section{Introduction}}
%\label{sec:intro}
Wave-particle resonant interactions play an important role  in a wide range of wave instabilities with frequency chirping in fusion and magnetoshperic plasmas.
In the fusion related plasmas, the frequency chirping often appears with the Alfv\'en wave instabilities \cite{chen2016,wang2018,wang2012,wang2012a,breizman2010,lilley2010} and energetic-particle-driven geodesic acoustic modes \cite{wang2013}.
The wave chirping is usually accompanied by fast ion loss and strong radial transport, which is greatly concerned in the fusion studies.
 Of particular interest in this work, 
the chorus wave
in the planetary magnetosphere
is a peculiar electromagnetic emission with frequency sweeping in the whistler-mode range of frequencies \cite{helliwell1965whistlers,burtis_magnetospheric_1976,tsurutani_postmidnight_1974}. 
The  magnetospheric chorus  plays an important role in the electron acceleration   in the Van Allen radiation belts \cite{horne_wave_2005,thorne_rapid_2013,reeves_electron_2013} and the pulsating aurora  and diffuse auroral precipitation \cite{nishimura_identifying_2010,kasahara_pulsating_2018,thorne_scattering_2010}.
The chorus emission is also related to  the studies of the superradiance in free-electron lasers \cite{zonca_nonlinear_2021, soto-chavez2012}.
Recently,
the chirping behavior similar to chorus waves has  been demonstrated in the laboratory plasmas \cite{vancompernolle2015,saitoh2024}.

The physics of chorus waves has been extensively studied as the nonlinear resonant wave-particle interactions 
% between the resonant trapped electrons and the unstable whistler waves from the electron temperature anisotropy
 \cite{omura_theory_2008, an2019, nunn_numerical_1990,tao_theoretical_2020,sudan_theory_1971,vomvoridis_nonlinear_1980,dawson1983,zheng2024,zonca_theoretical_2021}. % The Hamiltonian treatment of a particle nearly resonant with a monochromatic whistler wave in the inhomogeneous magnetic field has been extensively studied \cite{albert1993,albert2021}.
% The particle dynamics can be described by a modified pendulum Hamiltonian, where the trapped particles feel both the wave restoring force and the background drag force related to the frequency variation and magnetic field gradient \cite{zheng2024,tao_trap-release-amplify_2021}.
% If parameter changes are slow compared to the particle bounce motion, the system is adiabatic with an associated adiabatic invariant, similar to a pendulum system. 
% However,  recent studies  based on chirping and bounce scales suggest that  in  frequency chirping  chorus waves, the nonlinear chirping timescale is comparable to the bounce period, resulting in nonadiabatic interaction  \cite{tao2017a,tao2017b} and the  fast wave variations in triggering regions result in nonadiabatic processes 
% \cite{tao_trap-release-amplify_2021}. 
% Additionally,  the assumption of chirping adiabaticity  has been recently shown not to be valid, especially in the early phases of chirping  \cite{bierwage2021}.
% During chorus wave triggering and propagation, particle interactions vary, invalidating adiabatic motion in early chirping phases. 
% Despite this, stable trapped behavior exists away from the equator, where adiabaticity is poorly understood. 
% On the other hand, the adiabatic invariant governing resonant interaction with frequency chirping chorus waves remains undefined. 
% Establishing strict criteria for particle adiabaticity is crucial, aiding in identifying interaction stages and simplifying chorus evolution analysis.
Traditionally, the behavior of systems during particle bounce motion has been likened to an adiabatic process, drawing parallels with the  swing of a pendulum governed by an associated adiabatic invariant. However, recent research has catalyzed a significant shift in this understanding. 
Studies into the intricate interplay between chirping and bounce scales have uncovered a notable departure from adiabaticity in frequency chirping chorus waves
 \cite{tao2017a,tao2017b}. In this scenario, the nonlinear chirping timescale closely aligns with the bounce period, instigating nonadiabatic interactions.
  Furthermore, the rapid fluctuations in wave characteristics within triggering regions have emerged as pivotal factors driving nonadiabatic processes
\cite{tao_trap-release-amplify_2021}. 
Recent findings also challenge the assumption of chirping adiabaticity, especially in the early phases of chirping
\cite{bierwage2021}. 
This evolving comprehension challenges established assumptions, highlighting the subtle and complex nature of wave-particle interactions in the Earth's magnetosphere. 
As chorus waves excite and propagate, particle interactions dynamically shift, invalidating adiabatic motion during these initial stages. Despite this complexity, stable trapped behavior persists away from the equator, where adiabaticity remains poorly understood. Meanwhile, the adiabatic invariant governing resonant interaction with frequency chirping chorus waves remains elusive. Establishing the  criteria for particle adiabaticity is crucial, facilitating the identification of interaction stages and streamlining the analysis of chorus wave evolution.


Focusing on the chirping onset,  we  have  previously developed a Hamiltonian theory to investigate frequency chirping chorus waves within a local resonance frame \cite{zheng2024}  and devised a hybrid Eulerian–Lagrangian Vlasov simulation method that accurately reproduces chorus wave generation \cite{zheng2023b}. 
Since  the background magnetic field inhomogeneity  plays a critical role  in frequency chirping chorus waves \cite{wu2023,wu_controlling_2020}, the Vlasov simulations use  realistic parameters of the magnetosphere and closely mirror the satellite observations \cite{zheng2024,zheng2023b}, facilitating unique test particle simulations. 
In this work, our analysis of adiabaticity in magnetospheric chorus waves relies on test particle simulations to compute the adiabatic invariant, revealing variations in  adiabaticity of trapped particles across reference frames and spatial regions.  
Unlike conventional approaches \cite{huanghua_test_ptc,omura_test_ptc,tao_test_ptc}, which analyze particle dynamics within a laboratory frame with rapidly varying phase terms, we synchronize interactions with the resonance frame in this study. This allowed for adiabatic considerations while particles remain trapped \cite{CARY1989287}. 
Our simulation of resonant electron dynamics and chorus wave evolution affirm the validity of this adiabatic description. 
Additionally, we derive an analytic current expression, providing insights into the underlying physics of wave chirping. 
Finally, we validate our findings with the simplified model of energetic trapped particle current and the Vlasov simulation.

%We organize our 
%The paper is organized as follows. In section~\ref{sec:theory}, we review the Hamiltonian theory in the resonance reference frame. In section~\ref{sec:dis},  
%the adiabaticity 
%is examined by the test particle simulation and  the nonlinear current 
%is derived with adiabatic approximation. Finally, the summary is presented in section~\ref{sec:conc}.